\documentclass[a4paper, 12pt]{exam}

% Exam class is broken without this.
\makeatletter
\expandafter\providecommand\expandafter*\csname ver@framed.sty\endcsname
{2003/07/21 v0.8a Simulated by exam}
\makeatother

% Enables the use of colour.
\usepackage{xcolor}
% Syntax high-lighting for code. Requires Python's pygments.
\usepackage{minted}
% Enables the use of umlauts and other accents.
\usepackage[utf8]{inputenc}
% Diagrams.
\usepackage{tikz}
% Settings for captions, such as sideways captions.
\usepackage{caption}
% Symbols for units, like degrees and ohms.
\usepackage{gensymb}
% Latin modern fonts - better looking than the defaults.
\usepackage{lmodern}
% Allows for columns spanning multiple rows in tables.
\usepackage{multirow}
% Better looking tables, including nicer borders.
\usepackage{booktabs}
% More math symbols.
\usepackage{amssymb}
% More math layouts, equation arrays, etc.
\usepackage{amsmath}
% More math fonts, like mathbb.
\usepackage{amsfonts}
% More theorem environments.
\usepackage{amsthm}
% More column formats for tables.
\usepackage{array}
% Adjust the sizes of box environments.
\usepackage{adjustbox}
% Better looking single quotes in verbatim and minted environments.
\usepackage{upquote}
% URLs.
\usepackage[hidelinks]{hyperref}
% Better blank space decisions.
\usepackage{xspace}
% Better looking tikz trees.
\usepackage{forest}
% For plotting.
\usepackage{pgfplots}

% Various tikz libraries.
\usetikzlibrary{mindmap}
\usetikzlibrary{shadows}
\usetikzlibrary{arrows}
\usetikzlibrary{positioning}
\usetikzlibrary{chains}
\usetikzlibrary{fit}
\usetikzlibrary{shapes}
\usetikzlibrary{decorations.markings}
\usetikzlibrary{calc}
\usetikzlibrary{arrows.meta}

% GMIT colours.
\definecolor{gmitblue}{RGB}{20,134,225}
\definecolor{gmitred}{RGB}{220,20,60}
\definecolor{gmitgrey}{RGB}{67,67,67}

% Rename Bibliography to a smaller "Refereces".
\renewcommand{\refname}{\selectfont\normalsize References} 

% Stop minted high-lighting errors.
\makeatletter
\expandafter\def\csname PYGdefault@tok@err\endcsname{\def\PYGdefault@bc##1{{\strut ##1}}}
\makeatother

% Set the header and footer.
\pagestyle{headandfoot}
\header{\textbf{Problems: Python fundamentals}}{}{ian.mcloughlin@gmit.ie}
\footer{}{Page \thepage\ of \numpages}{}

% Change some things to do with marks.
\marksnotpoints
\pointsinrightmargin

% Empty cover page.
\begin{coverpages}
\end{coverpages}

% Print answers
\printanswers

\begin{document}

\noindent
The following exercises are related to the Python programming language~\cite{pythonwebsite}.

\begin{questions}

\question
Write a function \mintinline{python}{sumultiply} that takes two integer arguments and returns their product.
The function should not use the \mintinline{python}{*} or \mintinline{python}{/} operators.
For example:
\begin{minted}{python}
> sumultiply(11, 13)
143
> sumultiply(5, 123)
615
\end{minted}

\begin{solution}
  \inputminted{python}{solutions/sumultiply.py}
\end{solution}


\question
Write a function \mintinline{python}{ispalindrome} that takes a string and returns \mintinline{python}{True} if the string is a palindrome and \mintinline{python}{False} otherwise.
For example:
\begin{minted}{python}
> ispalindrome("radar")
True
> ispalindrome("radars")
False
\end{minted}

\begin{solution}
  \inputminted{python}{solutions/ispalindrome.py}
\end{solution}

 
\question
Write a function \mintinline{python}{simpleinterest} that, for a loan with simple interest, takes a principal amount, an interest rate, and a number of periods, and returns the total amount repaid. 
\begin{minted}{python}
> simpleinterest(1000, 3, 5)
1150.0
> simpleinterest(1000, 7, 10)
1700.0
\end{minted}

\begin{solution}
  \inputminted{python}{solutions/simpleinterest.py}
\end{solution}


\question
Write a function \mintinline{python}{compoundinterest} that, for a loan with compound interest, takes a principal amount, an interest rate, and a number of periods, and returns the total amount repaid. 
\begin{minted}{python}
> compoundinterest(1000, 3, 5)
1159.27
> compoundinterest(1000, 7, 10)
1967.15
\end{minted}

\begin{solution}
  \inputminted{python}{solutions/compoundinterest.py}
\end{solution}


\question
Write a function \mintinline{python}{newtonsroot} that takes a number \( x \) and returns its square root correct to six decimal places as calculated by Newton's method.
Newton's method is to make an initial (random) guess \( r_0 \) at the square root, and to repeatedly improve it as follows:
\[ r_{i+1} = r_i - \frac{r_i^2 - x}{2 r_i} \]
For example:
\begin{minted}{python}
> newtonsroot(100)
10.0
> newtonsroot(144)
12.0
\end{minted}

\begin{solution}
  \inputminted{python}{solutions/newtonsroot.py}
\end{solution}


\question
Write a function \mintinline{python}{pitondecs} that takes an integer \( n \) and returns \( \pi \) correct to \( n \) decimal places.
For example:
\begin{minted}{python}
> pitondecs(2)
3.14
> pitondecs(6)
3.141593
\end{minted}

\begin{solution}
  \inputminted{python}{solutions/pitondecs.py}
\end{solution}


\question
Write a function \mintinline{python}{etondecs} that takes an integer \( n \) and returns \( \epsilon \) correct to \( n \) decimal places.
For example:
\begin{minted}{python}
> etondecs(2)
2.72
> etondecs(6)
2.718282
\end{minted}

\begin{solution}
  \inputminted{python}{solutions/etondecs.py}
\end{solution}

\question
Write a function \mintinline{python}{caesar} that takes a string and an integer $n$ and returns the string with each letter shifted $n$ places in the alphabet.
For example:
\begin{minted}{python}
> caesar('abcd', 3)
'defg'
> caesar('Hello, world!', 2)
'Jgnnq, yqtnf!'
\end{minted}
 
\begin{solution}
  \inputminted{python}{solutions/caesar.py}
\end{solution}


\question
Write a function \mintinline{python}{sortlist} that takes a list of integers and returns a copy of it sorted.
Note that Python has a built-in sort function, but try to solve this problem without using it.
For example:
\begin{minted}{python}
> sortlist([3,1,2])
[1,2,3]
> sortlist([10,-9,5,-1,0])
[-9,-1,0,5,10]
\end{minted}

\begin{solution}
  \inputminted{python}{solutions/sortlist.py}
\end{solution}


\question
Write a function \mintinline{python}{countstr} that takes string and returns, for each character in the string, the number of times the character is contained in it.
You might use a dictionary for this purpose.
For example:
\begin{minted}{python}
> countstr('aaacbb')
{'a': 3,'c': 1,'b': 2}
> countstr('Hello, world!')
{'H': 1,'e': 1,'l': 3,'o': 2,',': 1,' ': 1,'w': 1,'r': 1,'d': 1,'!': 1}
\end{minted}

\begin{solution}
  \inputminted{python}{solutions/countstr.py}
\end{solution}


\end{questions}

\bibliographystyle{plain}
\bibliography{bibliography}
\end{document}
