\documentclass[a4paper, 12pt]{exam}

% Exam class is broken without this.
\makeatletter
\expandafter\providecommand\expandafter*\csname ver@framed.sty\endcsname
{2003/07/21 v0.8a Simulated by exam}
\makeatother

% Enables the use of colour.
\usepackage{xcolor}
% Syntax high-lighting for code. Requires Python's pygments.
\usepackage{minted}
% Enables the use of umlauts and other accents.
\usepackage[utf8]{inputenc}
% Diagrams.
\usepackage{tikz}
% Settings for captions, such as sideways captions.
\usepackage{caption}
% Symbols for units, like degrees and ohms.
\usepackage{gensymb}
% Latin modern fonts - better looking than the defaults.
\usepackage{lmodern}
% Allows for columns spanning multiple rows in tables.
\usepackage{multirow}
% Better looking tables, including nicer borders.
\usepackage{booktabs}
% More math symbols.
\usepackage{amssymb}
% More math layouts, equation arrays, etc.
\usepackage{amsmath}
% More math fonts, like mathbb.
\usepackage{amsfonts}
% More theorem environments.
\usepackage{amsthm}
% More column formats for tables.
\usepackage{array}
% Adjust the sizes of box environments.
\usepackage{adjustbox}
% Better looking single quotes in verbatim and minted environments.
\usepackage{upquote}
% URLs.
\usepackage[hidelinks]{hyperref}
% Better blank space decisions.
\usepackage{xspace}
% Better looking tikz trees.
\usepackage{forest}
% For plotting.
\usepackage{pgfplots}

% Various tikz libraries.
\usetikzlibrary{mindmap}
\usetikzlibrary{shadows}
\usetikzlibrary{arrows}
\usetikzlibrary{positioning}
\usetikzlibrary{chains}
\usetikzlibrary{fit}
\usetikzlibrary{shapes}
\usetikzlibrary{decorations.markings}
\usetikzlibrary{calc}
\usetikzlibrary{arrows.meta}

% GMIT colours.
\definecolor{gmitblue}{RGB}{20,134,225}
\definecolor{gmitred}{RGB}{220,20,60}
\definecolor{gmitgrey}{RGB}{67,67,67}

% Rename Bibliography to a smaller "Refereces".
\renewcommand{\refname}{\selectfont\normalsize References} 

% Stop minted high-lighting errors.
\makeatletter
\expandafter\def\csname PYGdefault@tok@err\endcsname{\def\PYGdefault@bc##1{{\strut ##1}}}
\makeatother

% Set the header and footer.
\pagestyle{headandfoot}
\header{\textbf{Problems: Python fundamentals}}{}{ian.mcloughlin@gmit.ie}
\footer{}{Page \thepage\ of \numpages}{}

% Change some things to do with marks.
\marksnotpoints
\pointsinrightmargin

% Empty cover page.
\begin{coverpages}
\end{coverpages}

% Print answers
\printanswers

\begin{document}

\noindent
The following exercises are related to the Python programming language~\cite{pythonwebsite}.

\begin{questions}

\question
Write a function \mintinline{python}{sumultiply} that takes two integer arguments and returns their product.
The function should not use the \mintinline{python}{*} or \mintinline{python}{/} operators.
For example:
\begin{minted}{scheme}
> sumultiply(11, 13)
143
> sumultiply(5, 123)
615
\end{minted}

\begin{solution}
  \begin{minted}{python}
True
  \end{minted}
\end{solution}

\question
Write a function \mintinline{python}{ispalindrome} that takes a string and returns \mintinline{python}{True} if the string is a palindrome and \mintinline{False} otherwise.
For example:
\begin{minted}{scheme}
> ispalindrome("radar")
True
> ispalindrome("radars")
False
\end{minted}

\begin{solution}
  \begin{minted}{python}
True
  \end{minted}
\end{solution}

\question
Write a function \mintinline{python}{newtonsroot} that takes a number $x$ and returns its square root correct to six decimal places as calculated by Newton's method.
Newton's method is to make an initial (random) guess $r_0$ at the square root, and to repeatedly improve it as follows:
\[ r_{i+1} = r_i - \frac{r_i^2 - x}{2 r_i} \]
For example:
\begin{minted}{scheme}
> newtonsroot(100)
10.0
> newtonsroot(144)
12.0
\end{minted}

\begin{solution}
  \begin{minted}{python}
True
  \end{minted}
\end{solution}



\end{questions}

\bibliographystyle{plain}
\bibliography{bibliography}
\end{document}
